%%%%%%%%%%%%%%%%%%%%%%%%%%%%%%%%%%%%%%%%%
% University/School Laboratory Report
% LaTeX Template
% Version 3.0 (4/2/13)
%
% This template has been downloaded from:
% http://www.LaTeXTemplates.com
%
% Original author:
% Linux and Unix Users Group at Virginia Tech Wiki 
% (https://vtluug.org/wiki/Example_LaTeX_chem_lab_report)
%
% License:
% CC BY-NC-SA 3.0 (http://creativecommons.org/licenses/by-nc-sa/3.0/)
%
%%%%%%%%%%%%%%%%%%%%%%%%%%%%%%%%%%%%%%%%%

%----------------------------------------------------------------------------------------
%	PACKAGES AND DOCUMENT CONFIGURATIONS
%----------------------------------------------------------------------------------------

\documentclass[11pt]{article}
\usepackage{amssymb, amsmath}
\usepackage{url}
\usepackage{graphicx}
\usepackage{courier}
%\usepackage{times} % Uncomment to use the Times New Roman font

%----------------------------------------------------------------------------------------
%	DOCUMENT INFORMATION
%----------------------------------------------------------------------------------------

\title{Parallel Memory Allocation \\ A Scalable Solution \\ CSC469} % Title

\date{\today} % Date for the report

\begin{document}

\maketitle % Insert the title, author and date

\begin{center}
\begin{tabular}{l r}
Partners: & Daniel Bloemendal \\ % Partner names
& Simon Scott \\
Instructor: & Professor Angela Demke Brown % Instructor/supervisor
\end{tabular}
\end{center}

% If you wish to include an abstract, uncomment the lines below
% \begin{abstract}
% Abstract text
% \end{abstract}

%----------------------------------------------------------------------------------------
%	SECTION 1
%----------------------------------------------------------------------------------------

\section{Motivation}

To create a scalable parallel memory allocator that is at least as fast as the standard c malloc.


\section{Hoard}

\indent \indent We choose to implement the Hoard Memory Allocator based on both its proven history and its utilization in modern Operating Systems. 
\\
\textbf{False Sharing}
\\
\indent Hoard utilizes Superblocks and multiplle heaps to avoid most active and passive false sharing. 
\\
\textbf{Blowup}
\\
\textbf{Scalability}

	In order to 
\section{Structures}

\textbf{Superblock}
\texttt{
\\
struct SUPERBLOCK\_T \{
\\
    heap\_t* heap;
\\
    uint8\_t group;
\\
    int size\_class;
\\
    size\_t block\_size;
\\
    size\_t block\_count;
\\
    size\_t block\_used;
\\
    blockptr\_t next\_block;
\\
    blockptr\_t next\_free;
\\
    superblock\_t\* prev;
\\
    superblock\_t\* next;
\\
\} \_\_attribute\_\_((aligned(ARCH\_CACHE\_ALIGNMENT)));
}
\\
\\


\section{Results}
\textbf{Speed}
\\
\textbf{Scalability}
\\
\section{Discussion}
 
%----------------------------------------------------------------------------------------
%	SECTION 2
%----------------------------------------------------------------------------------------


%----------------------------------------------------------------------------------------
%	BIBLIOGRAPHY
%----------------------------------------------------------------------------------------
\section{References}
\bibliographystyle{unsrt}
Berger, Emery. "Hoard: A Scalable Memory Allocator for Multithreaded Applications." UMass CS | School of Computer Science. N.p., n.d. Web. 11 Nov. 2013. <http://people.cs.umass.edu/>.
\bibliography{sample}
\\
\\
Berger, Emery. "Hoard: A Fast, Scalable, and Memory-Efficient Allocator for Shared-Memory Multiprocessors." Ftp://ftp.cs.utexas.edu. N.p., n.d. Web. <ftp://ftp.cs.utexas.edu/pub/techreports/tr99-22.pdf>.

%----------------------------------------------------------------------------------------


\end{document}