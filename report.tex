%%%%%%%%%%%%%%%%%%%%%%%%%%%%%%%%%%%%%%%%%
% University/School Laboratory Report
% LaTeX Template
% Version 3.0 (4/2/13)
%
% This template has been downloaded from:
% http://www.LaTeXTemplates.com
%
% Original author:
% Linux and Unix Users Group at Virginia Tech Wiki 
% (https://vtluug.org/wiki/Example_LaTeX_chem_lab_report)
%
% License:
% CC BY-NC-SA 3.0 (http://creativecommons.org/licenses/by-nc-sa/3.0/)
%
%%%%%%%%%%%%%%%%%%%%%%%%%%%%%%%%%%%%%%%%%

%----------------------------------------------------------------------------------------
%	PACKAGES AND DOCUMENT CONFIGURATIONS
%----------------------------------------------------------------------------------------

\documentclass[11pt]{article}
\usepackage{amssymb, amsmath}
\usepackage{url}
\usepackage{graphicx}
\usepackage{courier}
%\usepackage{times} % Uncomment to use the Times New Roman font

%----------------------------------------------------------------------------------------
%	DOCUMENT INFORMATION
%----------------------------------------------------------------------------------------

\title{Parallel Memory Allocation \\ A Scalable Solution \\ CSC469} % Title

\date{\today} % Date for the report

\begin{document}

\maketitle % Insert the title, author and date

\begin{center}
\begin{tabular}{l r}
Partners: & Daniel Bloemendal \\ % Partner names
& Simon Scott \\
Instructor: & Professor Angela Demke Brown % Instructor/supervisor
\end{tabular}
\end{center}

% If you wish to include an abstract, uncomment the lines below
% \begin{abstract}
% Abstract text
% \end{abstract}

%----------------------------------------------------------------------------------------
%	SECTION 1
%----------------------------------------------------------------------------------------

\section{Motivation}

To create a scalable parallel memory allocator that is at least as fast as the standard c malloc.


\section{Hoard Implementation}

\indent \indent We choose to implement the Hoard Memory Allocator based on both its proven history and its utilization in modern Operating Systems.  We use a global heap, and a CPU scaling factor which can be experimentally set to determine a number of per CPU heaps to initilize.  Threads are assigned to run on specific heaps.  The CPU scaling factor along with a more advanced hashing function help to prevent concurrent threads from being assigned to the same CPU.
\\
\newpage
\noindent
\textbf{Blowup}
\\
\indent  Utilizing an \textit{emptyness threshold} which can be experimentally set, we divide the Superblocks in a specific per processor heap by \textit{fullness groups}.  When a heap drops below the emptyness threshold, we transfer a superblock that is at least emptyness threshold empty.  By implementing this threshold, the same invarient that Berger used in his experiments in the original paper on Hoard holds (CITATION)
\\
\textbf{Fragmentation}
\\
\indent Inside Superblocks.  Superblocks use size classes for internal blocks which are powers of \textit{b}, and limit internal fragmentation to \textit{b} (CITATION) pp\#2. \textit{b} can be experimentally set.
\\
\indent Outside Superblocks.  Superblocks are not returned to the Operating System, they are \textit{recycled} for re use by any size class.
\\
\textbf{False Sharing}
\\
\indent Hoard utilizes Superblocks and multiplle heaps to avoid most active and passive false sharing. 



\section{Structures}

\textbf{Superblock}
\texttt{
\\
struct SUPERBLOCK\_T \{
\\
    heap\_t* heap;
\\
    uint8\_t group;
\\
    int size\_class;
\\
    size\_t block\_size;
\\
    size\_t block\_count;
\\
    size\_t block\_used;
\\
    blockptr\_t next\_block;
\\
    blockptr\_t next\_free;
\\
    superblock\_t\* prev;
\\
    superblock\_t\* next;
\\
\} \_\_attribute\_\_((aligned(ARCH\_CACHE\_ALIGNMENT)));
}
\\
\\
\newpage

\noindent \textbf{Heap}
\\
\texttt{struct HEAP\_T \{
\\
\indent    int index;
\\
\indent    size\_t mem\_used;
\\
 \indent   size\_t mem\_allocated;
\\
 \indent   superblock\_t\* bins[ALLOC\_HOARD\_FULLNESS\_GROUPS];
\\
\};}


\section{Results}
\textbf{Speed}
\\
\textbf{Scalability}
\\
\section{Discussion}
 
%----------------------------------------------------------------------------------------
%	SECTION 2
%----------------------------------------------------------------------------------------


%----------------------------------------------------------------------------------------
%	BIBLIOGRAPHY
%----------------------------------------------------------------------------------------
\section{References}
\bibliographystyle{unsrt}
Berger, Emery. "Hoard: A Scalable Memory Allocator for Multithreaded Applications." UMass CS | School of Computer Science. N.p., n.d. Web. 11 Nov. 2013. <http://people.cs.umass.edu/>.
\bibliography{sample}
\\
\\
Berger, Emery. "Hoard: A Fast, Scalable, and Memory-Efficient Allocator for Shared-Memory Multiprocessors." Ftp://ftp.cs.utexas.edu. N.p., n.d. Web. <ftp://ftp.cs.utexas.edu/pub/techreports/tr99-22.pdf>.

%----------------------------------------------------------------------------------------


\end{document}